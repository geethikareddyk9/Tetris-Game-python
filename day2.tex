\documentclass[14pt]{beamer}
\title{TETRIS}
\date{Date: 21-10-2021}
\author[Bvrith]{Haripriya      20WH1A0532  CSE \\ Sreekruthi K  20WH1A05D4 CSE \\ Geethika Reddy K  20WH1A1270  IT \\ Maithreyi  20WH1A0403  ECE \\ Varshitha  20WH1A0461  ECE }
\usefonttheme{serif}
\usepackage{bookman}
\usepackage{hyperref}
\usepackage[T1]{fontenc}
\usepackage{graphicx}
\usecolortheme{orchid}
\beamertemplateballitem

\begin{document}
    \begin{frame}
        \titlepage
    \end{frame}

\begin{frame}{Description}
Tetris is a video game developed in 1984. The game has a simple goal of destroying lines of block before it reaches the top.The line is made up of a square block.



\vskip 1cm
\begin{block}{Rules}
The goal is to drop blocks, called tetrominoes, down into a playing field to make lines.
The Tetris game requires players to strategically rotate, move, and drop a procession of Tetriminos that fall into the rectangular Matrix at increasing speeds.

\end{block}

\end{frame}
\begin{frame}

\frametitle{Learnings}

\begin{itemize}

\item LateX - \\ is a free software created to make that easier to produce general purpose books and articles within TeX. It is present in linux OS. 
\item GIT
\item Pygame

\end{itemize}

\end{frame}

    \begin{frame}
	\frametitle{Approach}
	\begin{itemize}
	    \item Creating the graphics for the game
           \item Using matrix rotation in moving the blocks
           \item Adding points to each right move
           \item Level up after achieving the given target
	\end{itemize}
    \end{frame}
    
    \begin{frame}
	\frametitle{Challenges}
        \begin{itemize}
	    \item Learning Pygame documentation to use it in the code
            \item Getting familiar with Latex and Git lab
        \end{itemize}
    \end{frame}
    \begin{frame}
	\frametitle{Resources/Reference}
	\begin{itemize}
		\item Pygame
                 \item PYcharm
                 \item Windows IDEL SHELL
	\end{itemize}
    \end{frame}
    \begin{frame}
	\frametitle{Individual Progress}
        \begin{itemize}
	     \item Learning pygame and building graphics - Haripriya, Sreekruthi
	     \item Framing an algorithm to build the game - Maithreyi
             \item Working on latex presentation, git - Geetika
        \end{itemize}
    \end{frame}
    
    \begin{frame}
	\begin{center}
	     END (or) THANK YOU
	\end{center}
    \end{frame}
\end{document}
